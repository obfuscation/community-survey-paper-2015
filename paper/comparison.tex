%%%%%%%%%%%%%%%%%%%%%%%%%%%%%%%%%%%%%%%%%%%%%%%%%%%%%%%%%%%%%%%%%%%%%%%%%%%%%%%%
% Community-driven Obfuscation Paper
%
% This is comparison.tex. 
%%%%%%%%%%%%%%%%%%%%%%%%%%%%%%%%%%%%%%%%%%%%%%%%%%%%%%%%%%%%%%%%%%%%%%%%%%%%%%%%
\section{Comparative Overview of Existing Obfuscation Techniques}
\label{section:comparison} 

\subsection{Obfuscation Approaches}

\paragraph{Randomization techniques.}
The ``look like nothing'' approach, e.g., obfsproxy and Dust. 

\paragraph{Impersonation techniques.}

\subparagraph{Parroting.} 	Roll own implementation to make network traffic
approximately look like something else. Examples include Stegotorus, FTE,
others.

\subparagraph{Hide-within techniques.}
\tcrnote{In my mind this translates to ``tunneling''. Why not use tunneling
instead? We can also discuss the need for some parroting potentially of the
tunneled content.}

\subparagraph{Decoy routing.} 
A special class of tunneling protocols that also cleverly use redirection. An
example is meek. 

%they work at the ``transport'' layer)}

\tcrnote{Any other broad classes of obfuscation we are missing?}


\subsection{Implementations}

\tcrnote{An overview of the implementations out there }

