%%%%%%%%%%%%%%%%%%%%%%%%%%%%%%%%%%%%%%%%%%%%%%%%%%%%%%%%%%%%%%%%%%%%%%%%%%%%%%%%
% Community-driven Obfuscation Paper
%
% This is comparison.tex. 
%%%%%%%%%%%%%%%%%%%%%%%%%%%%%%%%%%%%%%%%%%%%%%%%%%%%%%%%%%%%%%%%%%%%%%%%%%%%%%%%
\section{Comparative Overview of Existing Obfuscation Techniques}
\label{section:comparison} 

\subsection{Obfuscation Approaches}

\paragraph{Randomization techniques.}

Randomization techniques aim to defeat DPI machines that detect
signatures on the content of the communication. These ``look like
nothing'' transports bypass censorship by encrypting the application
layer content to make it look like random bytes. These transports
might also use padding or timing obfuscation to bypass DPI boxes that
can detect based on session metadata. The downside of these transports
is that they are trivially blocked by whitelists.

\tcrnoteg{Do we want to overlink? We could link to DUST, obfuscated-ssh,
obfsproxy, scramblesuit, etc. here. Or maybe we should link in the
implementations section?}

\paragraph{Impersonation techniques.}

\subparagraph{Parroting.} 	Roll own implementation to make network traffic
approximately look like something else. Examples include Stegotorus, FTE,
others.

\subparagraph{Hide-within techniques.}
\tcrnote{In my mind this translates to ``tunneling''. Why not use tunneling
instead? We can also discuss the need for some parroting potentially of the
tunneled content.}

\subparagraph{Decoy routing.} 
A special class of tunneling protocols that also cleverly use redirection. An
example is meek. 

%they work at the ``transport'' layer)}

\tcrnote{Any other broad classes of obfuscation we are missing?}


\subsection{Implementations}

\tcrnote{An overview of the implementations out there.}

\tcrnoteg{Maybe also a table with the above categories and how implementation fit to each category.}

